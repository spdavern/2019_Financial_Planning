\documentclass[]{article}
\usepackage{lmodern}
\usepackage{amssymb,amsmath}
\usepackage{ifxetex,ifluatex}
\usepackage{fixltx2e} % provides \textsubscript
\ifnum 0\ifxetex 1\fi\ifluatex 1\fi=0 % if pdftex
  \usepackage[T1]{fontenc}
  \usepackage[utf8]{inputenc}
\else % if luatex or xelatex
  \ifxetex
    \usepackage{mathspec}
  \else
    \usepackage{fontspec}
  \fi
  \defaultfontfeatures{Ligatures=TeX,Scale=MatchLowercase}
\fi
% use upquote if available, for straight quotes in verbatim environments
\IfFileExists{upquote.sty}{\usepackage{upquote}}{}
% use microtype if available
\IfFileExists{microtype.sty}{%
\usepackage{microtype}
\UseMicrotypeSet[protrusion]{basicmath} % disable protrusion for tt fonts
}{}
\usepackage[margin=1in]{geometry}
\usepackage{hyperref}
\hypersetup{unicode=true,
            pdftitle={Transferring 2013 Analysis to R},
            pdfauthor={Sean Davern},
            pdfborder={0 0 0},
            breaklinks=true}
\urlstyle{same}  % don't use monospace font for urls
\usepackage{color}
\usepackage{fancyvrb}
\newcommand{\VerbBar}{|}
\newcommand{\VERB}{\Verb[commandchars=\\\{\}]}
\DefineVerbatimEnvironment{Highlighting}{Verbatim}{commandchars=\\\{\}}
% Add ',fontsize=\small' for more characters per line
\usepackage{framed}
\definecolor{shadecolor}{RGB}{248,248,248}
\newenvironment{Shaded}{\begin{snugshade}}{\end{snugshade}}
\newcommand{\AlertTok}[1]{\textcolor[rgb]{0.94,0.16,0.16}{#1}}
\newcommand{\AnnotationTok}[1]{\textcolor[rgb]{0.56,0.35,0.01}{\textbf{\textit{#1}}}}
\newcommand{\AttributeTok}[1]{\textcolor[rgb]{0.77,0.63,0.00}{#1}}
\newcommand{\BaseNTok}[1]{\textcolor[rgb]{0.00,0.00,0.81}{#1}}
\newcommand{\BuiltInTok}[1]{#1}
\newcommand{\CharTok}[1]{\textcolor[rgb]{0.31,0.60,0.02}{#1}}
\newcommand{\CommentTok}[1]{\textcolor[rgb]{0.56,0.35,0.01}{\textit{#1}}}
\newcommand{\CommentVarTok}[1]{\textcolor[rgb]{0.56,0.35,0.01}{\textbf{\textit{#1}}}}
\newcommand{\ConstantTok}[1]{\textcolor[rgb]{0.00,0.00,0.00}{#1}}
\newcommand{\ControlFlowTok}[1]{\textcolor[rgb]{0.13,0.29,0.53}{\textbf{#1}}}
\newcommand{\DataTypeTok}[1]{\textcolor[rgb]{0.13,0.29,0.53}{#1}}
\newcommand{\DecValTok}[1]{\textcolor[rgb]{0.00,0.00,0.81}{#1}}
\newcommand{\DocumentationTok}[1]{\textcolor[rgb]{0.56,0.35,0.01}{\textbf{\textit{#1}}}}
\newcommand{\ErrorTok}[1]{\textcolor[rgb]{0.64,0.00,0.00}{\textbf{#1}}}
\newcommand{\ExtensionTok}[1]{#1}
\newcommand{\FloatTok}[1]{\textcolor[rgb]{0.00,0.00,0.81}{#1}}
\newcommand{\FunctionTok}[1]{\textcolor[rgb]{0.00,0.00,0.00}{#1}}
\newcommand{\ImportTok}[1]{#1}
\newcommand{\InformationTok}[1]{\textcolor[rgb]{0.56,0.35,0.01}{\textbf{\textit{#1}}}}
\newcommand{\KeywordTok}[1]{\textcolor[rgb]{0.13,0.29,0.53}{\textbf{#1}}}
\newcommand{\NormalTok}[1]{#1}
\newcommand{\OperatorTok}[1]{\textcolor[rgb]{0.81,0.36,0.00}{\textbf{#1}}}
\newcommand{\OtherTok}[1]{\textcolor[rgb]{0.56,0.35,0.01}{#1}}
\newcommand{\PreprocessorTok}[1]{\textcolor[rgb]{0.56,0.35,0.01}{\textit{#1}}}
\newcommand{\RegionMarkerTok}[1]{#1}
\newcommand{\SpecialCharTok}[1]{\textcolor[rgb]{0.00,0.00,0.00}{#1}}
\newcommand{\SpecialStringTok}[1]{\textcolor[rgb]{0.31,0.60,0.02}{#1}}
\newcommand{\StringTok}[1]{\textcolor[rgb]{0.31,0.60,0.02}{#1}}
\newcommand{\VariableTok}[1]{\textcolor[rgb]{0.00,0.00,0.00}{#1}}
\newcommand{\VerbatimStringTok}[1]{\textcolor[rgb]{0.31,0.60,0.02}{#1}}
\newcommand{\WarningTok}[1]{\textcolor[rgb]{0.56,0.35,0.01}{\textbf{\textit{#1}}}}
\usepackage{graphicx,grffile}
\makeatletter
\def\maxwidth{\ifdim\Gin@nat@width>\linewidth\linewidth\else\Gin@nat@width\fi}
\def\maxheight{\ifdim\Gin@nat@height>\textheight\textheight\else\Gin@nat@height\fi}
\makeatother
% Scale images if necessary, so that they will not overflow the page
% margins by default, and it is still possible to overwrite the defaults
% using explicit options in \includegraphics[width, height, ...]{}
\setkeys{Gin}{width=\maxwidth,height=\maxheight,keepaspectratio}
\IfFileExists{parskip.sty}{%
\usepackage{parskip}
}{% else
\setlength{\parindent}{0pt}
\setlength{\parskip}{6pt plus 2pt minus 1pt}
}
\setlength{\emergencystretch}{3em}  % prevent overfull lines
\providecommand{\tightlist}{%
  \setlength{\itemsep}{0pt}\setlength{\parskip}{0pt}}
\setcounter{secnumdepth}{0}
% Redefines (sub)paragraphs to behave more like sections
\ifx\paragraph\undefined\else
\let\oldparagraph\paragraph
\renewcommand{\paragraph}[1]{\oldparagraph{#1}\mbox{}}
\fi
\ifx\subparagraph\undefined\else
\let\oldsubparagraph\subparagraph
\renewcommand{\subparagraph}[1]{\oldsubparagraph{#1}\mbox{}}
\fi

%%% Use protect on footnotes to avoid problems with footnotes in titles
\let\rmarkdownfootnote\footnote%
\def\footnote{\protect\rmarkdownfootnote}

%%% Change title format to be more compact
\usepackage{titling}

% Create subtitle command for use in maketitle
\providecommand{\subtitle}[1]{
  \posttitle{
    \begin{center}\large#1\end{center}
    }
}

\setlength{\droptitle}{-2em}

  \title{Transferring 2013 Analysis to R}
    \pretitle{\vspace{\droptitle}\centering\huge}
  \posttitle{\par}
    \author{Sean Davern}
    \preauthor{\centering\large\emph}
  \postauthor{\par}
      \predate{\centering\large\emph}
  \postdate{\par}
    \date{June 5, 2019}


\begin{document}
\maketitle

\hypertarget{objective-of-this-work}{%
\section{Objective of this Work}\label{objective-of-this-work}}

The objective of the work captured in this notebook is to translate part
of an analysis done in 2013 (See Davern (2013)) into R to learn and
demonstrate multiple R capabilities and workflows.

\hypertarget{import-of-data}{%
\section{Import of Data}\label{import-of-data}}

The original data was provided by John Earling in a work book entitled
`Weekly PayPay \& Tithes .xls' workbook. The layout/format of that
workbook was not conducive to easily loading into R {[}nor JMP
originally{]} and so was transcribed into the workbook `Giving
Data.xlsx' and then read into R.

\begin{Shaded}
\begin{Highlighting}[]
\KeywordTok{source}\NormalTok{(}\StringTok{"../code/0-Extract data from Excel.R"}\NormalTok{)}
\NormalTok{df  }\CommentTok{# This is the data frame resulting from the import.}
\end{Highlighting}
\end{Shaded}

\begin{verbatim}
## # A tibble: 469 x 7
##    week.ending         month   year paypal offering  total monthly.giving.~
##    <dttm>              <chr>  <dbl>  <dbl>    <dbl>  <dbl>            <dbl>
##  1 2010-01-03 00:00:00 Janua~  2010     75    1560   1635                NA
##  2 2010-01-10 00:00:00 Janua~  2010    575    3129   3704                NA
##  3 2010-01-17 00:00:00 Janua~  2010    475    2025   2500                NA
##  4 2010-01-24 00:00:00 Janua~  2010     75    1180.  1255.               NA
##  5 2010-01-31 00:00:00 Janua~  2010   2180    2967.  5147.               45
##  6 2010-02-07 00:00:00 Febru~  2010     75    4722.  4798.               NA
##  7 2010-02-14 00:00:00 Febru~  2010    585    2925   3510                NA
##  8 2010-02-21 00:00:00 Febru~  2010    200    3299   3499                NA
##  9 2010-02-28 00:00:00 Febru~  2010   6350    4281  10631                41
## 10 2010-03-07 00:00:00 March   2010    770    1170   1940                NA
## # ... with 459 more rows
\end{verbatim}

\hypertarget{some-minor-data-validation}{%
\section{Some Minor Data Validation}\label{some-minor-data-validation}}

As a first validation I'll check that the weekly PayPal and offering
amounts sum to the weekly totals
\((paypal_i+{offering}_i\overset{?}=total_i)\), reporting only those
that aren't equal:

\begin{Shaded}
\begin{Highlighting}[]
\KeywordTok{source}\NormalTok{(}\StringTok{"../code/1-Validate totals.R"}\NormalTok{)}
\end{Highlighting}
\end{Shaded}

\begin{verbatim}
## ***** WARNING *****
\end{verbatim}

\begin{verbatim}
## Some 'total' observations don't equal the sum of 'paypal' and 'offering'!
\end{verbatim}

\begin{verbatim}
## # A tibble: 3 x 8
##   week.ending         month    year paypal offering total calcd.total  diff
##   <dttm>              <chr>   <dbl>  <dbl>    <dbl> <dbl>       <dbl> <dbl>
## 1 2015-12-27 00:00:00 Decemb~  2015   1902     9306 16958       11208  5750
## 2 2016-12-25 00:00:00 Decemb~  2016   4635    10089 22849       14724  8125
## 3 2017-04-23 00:00:00 April    2017    635     4480  4615        5115  -500
\end{verbatim}

Ok, so December 2015 and 2016 seem to have totals greater than accounted
for by the PayPal and offering amounts. That's perhaps explainable by
other end-of-year giving coming in another way. However, the April 2017
discrepancy seems to be missing \$500. I'll need to look into that.

\hypertarget{data-transformation}{%
\section{Data Transformation}\label{data-transformation}}

Aggregating the monthly totals and preparing to model month
values\ldots{}

\begin{Shaded}
\begin{Highlighting}[]
\CommentTok{# Data transformation: Calculate monthly giving totals.}
\CommentTok{# Make Month a categorical variable with levels in the order that}
\CommentTok{# months occur in the year otherwise months are sorted alphabetically.}
\NormalTok{df}\OperatorTok{$}\NormalTok{month <-}\StringTok{ }\KeywordTok{factor}\NormalTok{(df}\OperatorTok{$}\NormalTok{month, month.name)}
\CommentTok{# Aggregate the monthly Totals from giving.data in sums for each month.}
\NormalTok{MonthTotals <-}\StringTok{ }
\StringTok{  }\KeywordTok{aggregate}\NormalTok{(df}\OperatorTok{$}\NormalTok{total, }\DataTypeTok{by =} \KeywordTok{list}\NormalTok{(df}\OperatorTok{$}\NormalTok{month, df}\OperatorTok{$}\NormalTok{year), }\DataTypeTok{FUN =}\NormalTok{ sum)}
\CommentTok{# Exclude the months that don't have totals yet.}
\NormalTok{MonthTotals <-}\StringTok{ }\NormalTok{MonthTotals[}\KeywordTok{complete.cases}\NormalTok{(MonthTotals), ]}
\CommentTok{# Extract only rows containing 'monthly.giving.families' data.}
\NormalTok{df <-}\StringTok{ }\NormalTok{df[}\OperatorTok{!}\KeywordTok{is.na}\NormalTok{(df}\OperatorTok{$}\NormalTok{monthly.giving.families),]}
\CommentTok{# Now replace Totals (which were weekly totals) with calculated aggregates}
\NormalTok{df}\OperatorTok{$}\NormalTok{total <-}\StringTok{ }\NormalTok{MonthTotals}\OperatorTok{$}\NormalTok{x}
\CommentTok{# paypal & offering columns are now misleading (only week's value) so remove them.}
\NormalTok{df <-}\StringTok{ }\KeywordTok{select}\NormalTok{(df, }\OperatorTok{-}\NormalTok{paypal, }\OperatorTok{-}\NormalTok{offering) }
\end{Highlighting}
\end{Shaded}

Adding the number of giving Sundays in the month and the average giving
each week per month\ldots{}

\begin{Shaded}
\begin{Highlighting}[]
\KeywordTok{library}\NormalTok{(}\StringTok{"magrittr"}\NormalTok{)}
\KeywordTok{source}\NormalTok{(}\StringTok{"../code/NumOfGivenDayOfWeekInMonth.R"}\NormalTok{)}
\CommentTok{# Calculate and add the columns SundaysInMonth with calculated values}
\CommentTok{# and MonthsGivingPerWeek}
\NormalTok{df <-}\StringTok{  }\NormalTok{df }\OperatorTok
\StringTok{  }\KeywordTok{mutate}\NormalTok{(}\DataTypeTok{SundaysInMonth =}
           \KeywordTok{NumOfGivenDayOfWeekInMonth}\NormalTok{(df}\OperatorTok{$}\NormalTok{week.ending, }\StringTok{"Sunday"}\NormalTok{)) }\OperatorTok
\StringTok{  }\KeywordTok{mutate}\NormalTok{(}\DataTypeTok{MonthsGivingPerWeek =}\NormalTok{ total }\OperatorTok{/}\StringTok{ }\NormalTok{SundaysInMonth)}
\end{Highlighting}
\end{Shaded}

Enable modeling year as factor rather than a number\ldots{}

\begin{Shaded}
\begin{Highlighting}[]
\CommentTok{# Make year a categorical variable so coefficients are easier to interpret.}
\NormalTok{df}\OperatorTok{$}\NormalTok{year <-}\StringTok{ }\KeywordTok{as.factor}\NormalTok{(df}\OperatorTok{$}\NormalTok{year)}
\end{Highlighting}
\end{Shaded}

Save the resulting R tibble:

\begin{Shaded}
\begin{Highlighting}[]
\CommentTok{# Code chunk eval=false so files don't get overwritten willy nilly.}
\CommentTok{# Write it as a csv:}
\KeywordTok{write.csv}\NormalTok{(}\DataTypeTok{x =}\NormalTok{ df,}
          \DataTypeTok{file =} \StringTok{"../data/Cleaned and Transformed Giving Data.csv"}\NormalTok{,}
          \DataTypeTok{row.names =} \OtherTok{FALSE}\NormalTok{)}
\CommentTok{# Save it also as an R object that can be loaded into a new R object.}
\KeywordTok{saveRDS}\NormalTok{(df, }\DataTypeTok{file =} \StringTok{"../data/Cleaned and Transformed Giving Data.rds"}\NormalTok{)}
\end{Highlighting}
\end{Shaded}

\newpage

\hypertarget{replicating-previous-modeling}{%
\section{Replicating Previous
Modeling}\label{replicating-previous-modeling}}

The relatively simple model derived in 2013 (see Davern 2013, pg. 11)
and used again in 2018 used this model:
\[\text{Monthly Giving} = a_1+b_{year}+c_{month}\] where \(a_1\) is an
overall grand average of the monthly giving amount, \(b_{year}\) is an
adjustment for the given year and \(c_{month}\) is an adjustment for the
month. The model was originally regressed on giving data from Jan 2010
through August 2013 excluding 3 high-fliers with known exceptional
donations. We can now regress this model:

\begin{Shaded}
\begin{Highlighting}[]
\KeywordTok{library}\NormalTok{(}\StringTok{"magrittr"}\NormalTok{)}
\CommentTok{# Pair the data down to that used in the original analysis}
\NormalTok{df2 <-}\StringTok{ }\NormalTok{df[}\KeywordTok{as.Date}\NormalTok{(df}\OperatorTok{$}\NormalTok{week.ending) }\OperatorTok{>}\StringTok{ "2010-01-01"} \OperatorTok{&}\StringTok{ }
\StringTok{            }\KeywordTok{as.Date}\NormalTok{(df}\OperatorTok{$}\NormalTok{week.ending) }\OperatorTok{<}\StringTok{ "2013-08-31"}\NormalTok{,] }\OperatorTok
\StringTok{  }\KeywordTok{mutate}\NormalTok{(}\DataTypeTok{excluded =} \OtherTok{FALSE}\NormalTok{)}
\NormalTok{df2}\OperatorTok{$}\NormalTok{excluded[}\KeywordTok{as.Date}\NormalTok{(df2}\OperatorTok{$}\NormalTok{week.ending) }\OperatorTok{==}\StringTok{ "2010-02-28"}\NormalTok{] <-}\StringTok{ }\OtherTok{TRUE}
\NormalTok{df2}\OperatorTok{$}\NormalTok{excluded[}\KeywordTok{as.Date}\NormalTok{(df2}\OperatorTok{$}\NormalTok{week.ending) }\OperatorTok{==}\StringTok{ "2012-04-29"}\NormalTok{] <-}\StringTok{ }\OtherTok{TRUE}
\NormalTok{df2}\OperatorTok{$}\NormalTok{excluded[}\KeywordTok{as.Date}\NormalTok{(df2}\OperatorTok{$}\NormalTok{week.ending) }\OperatorTok{==}\StringTok{ "2012-12-30"}\NormalTok{] <-}\StringTok{ }\OtherTok{TRUE}
\CommentTok{# Regress the model cluding the indicated values:}
\NormalTok{mod <-}\StringTok{ }\KeywordTok{lm}\NormalTok{(}
  \DataTypeTok{formula =}\NormalTok{ total }\OperatorTok{~}
\StringTok{    }\NormalTok{year }\OperatorTok{+}\StringTok{ }\NormalTok{month,}
  \DataTypeTok{data =}\NormalTok{ df2[df2}\OperatorTok{$}\NormalTok{excluded}\OperatorTok{!=}\OtherTok{TRUE}\NormalTok{,]}
\NormalTok{)}
\end{Highlighting}
\end{Shaded}

Which gives the resulting model fit:

\includegraphics{Transfer_of_2013_Analysis_files/figure-latex/unnamed-chunk-6-1.pdf}

Note: excluded points {[}high fliers{]} are shown (above and below)
though they weren't included in the regression. Here are the fit
diagnostics:

\begin{verbatim}
## Analysis of Variance Table
## 
## Response: total
##           Df    Sum Sq Mean Sq F value Pr(>F)
## year       3  21431854 7143951  1.4506  0.251
## month     11  93473089 8497554  1.7254  0.123
## Residuals 26 128049780 4924992
\end{verbatim}

\includegraphics{Transfer_of_2013_Analysis_files/figure-latex/unnamed-chunk-7-1.pdf}

\begin{verbatim}
## 
## Call:
## lm(formula = total ~ year + month, data = df2[df2$excluded != 
##     TRUE, ])
## 
## Residuals:
##     Min      1Q  Median      3Q     Max 
## -3262.7 -1266.4   269.4  1181.2  4604.5 
## 
## Coefficients:
##                Estimate Std. Error t value Pr(>|t|)    
## (Intercept)    12539.40    1264.30   9.918 2.52e-10 ***
## year2011         708.72     933.26   0.759   0.4544    
## year2012        1531.73     997.47   1.536   0.1367    
## year2013        2295.36    1082.20   2.121   0.0436 *  
## monthFebruary   -210.59    1706.96  -0.123   0.9028    
## monthMarch      2164.52    1569.23   1.379   0.1795    
## monthApril       376.99    1707.70   0.221   0.8270    
## monthMay         752.68    1569.23   0.480   0.6355    
## monthJune       1195.27    1569.23   0.762   0.4531    
## monthJuly       2886.33    1569.23   1.839   0.0773 .  
## monthAugust      158.54    1569.23   0.101   0.9203    
## monthSeptember    37.28    1710.48   0.022   0.9828    
## monthOctober    4547.45    1710.48   2.659   0.0132 *  
## monthNovember  -1306.57    1710.48  -0.764   0.4518    
## monthDecember   2621.14    1956.39   1.340   0.1919    
## ---
## Signif. codes:  0 '***' 0.001 '**' 0.01 '*' 0.05 '.' 0.1 ' ' 1
## 
## Residual standard error: 2219 on 26 degrees of freedom
## Multiple R-squared:  0.4729, Adjusted R-squared:  0.1892 
## F-statistic: 1.666 on 14 and 26 DF,  p-value: 0.1261
\end{verbatim}

The results obtained are different in a number of ways from what was
obtained in 2013. The general fit (residuals) and shape of the predicted
values are similar to that obtained in 2013 though the June predictions
seem further away than the other months. The regressed coefficients are
obviously very different, but this is largely due to the method JMP uses
for regressing factor coefficients. However, the difference in the June
predictions and slightly different factor p-values tells me that
something in the underlying data is probably different. I'm not going to
spend the time to diagnose the precise details of the difference since
the objective is to translate the analysis to R rather than reproduce
it.

\hypertarget{references}{%
\section*{References}\label{references}}
\addcontentsline{toc}{section}{References}

\hypertarget{refs}{}
\leavevmode\hypertarget{ref-davern2013}{}%
Davern, Sean. 2013. ``An Assessment of Requested Seed Core Support for
Expansion Based on Analysis of Giving.'' \emph{Internal Report}, August.
\url{file://../Reports/August\%202013\%20Analysis\%20of\%20Giving.pdf}.


\end{document}
